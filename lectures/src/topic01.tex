%Paul E. West

%\documentclass[xcolor=svgnames]{beamer}
\documentclass{beamer}
\usepackage[boxed,vlined,figure]{algorithm2e}

%\usecolortheme[named=FireBrick]{structure}
%\usecolortheme[named=black]{structure}
%\usecolortheme{beetle}
%\usecolortheme{beaver}
%\usecolortheme{crane}
%\usecolortheme{dolphin}
%\usecolortheme{dove}
%\usecolortheme{fly}
%\usecolortheme{lily}
\usecolortheme{orchid}
%\usecolortheme{rose}
%\setbeamercolor{background canvas}{bg=Gold!25}
%\setbeamercolor{background canvas}{bg=Black!100}
%\setbeamercolor{foreground}{bg=Gold!25}
%\setbeamercolor{normal text}{fg=green,bg=black}
%\setbeamercolor*{palette primary}{use=structure,fg=green,bg=black}

\mode<presentation>{
    \usetheme{Darmstadt}
    \setbeamercovered{invisible}
    %\setbeamercovered{transparent}
    \setbeamercolor*{palette primary}{use=structure,fg=white,bg=blue}
    \setbeamercolor*{palette secondary}{use=structure,fg=white,bg=blue}
    \setbeamercolor*{palette tertiary}{use=structure,fg=white,bg=blue}
}

\usepackage[english]{babel}
\usepackage[latin1]{inputenc}
\usepackage{times}
\usepackage[T1]{fontenc}
%\usepackage{epsfig}
\usepackage{ulem}
\usepackage{color,soul}

\usepackage{graphicx}
\usepackage{amssymb}
\usepackage{url,hyperref}
\definecolor{beamer@blendedblue}{rgb}{1,.6,.2}
%\usepackage{tikz}
%\usetikzlibrary{shapes}
%\usetikzlibrary{arrows}
%\tikzstyle{block}=[draw opacity=0.7, line width=1.4cm]
\usepackage{listings}
\lstset{language=Java}
\lstset{showspaces=false}
\lstset{showstringspaces=false}
\lstset{tabsize=4}
\lstset{basicstyle=\tiny}


%\usecolortheme[overlystylish]{albatross}
%\usecolortheme[]{lily}
%\usecolortheme[]{albatross}
%\usecolortheme[]{orchid}
%\setbeamercolor{normal text}{fg=green!10}

\title{COIN 325: Java I and Elementary Data Structures}
\author{Dr. Paul E. West}

\institute{
  Department of Computer Science\\
  Charleston Southern University
}

\date{January 16, 2015}

\subject{Software Programming}
%\keywords{Performance Counters, Multicore}

%\pgfdeclareimage[height=1.0cm]{university-logo}{../imgs/csu-logo}
\pgfdeclareimage[height=0.75cm]{university-logo}{../imgs/csu-logo}
%\pgfdeclareimage[height=0.50cm]{university-logo}{../imgs/csu-logo}
\logo{\pgfuseimage{university-logo}}

\begin{document}

\begin{frame}
  \titlepage
\end{frame}

\section{First Day}
\subsection{}
\begin{frame}{About the Professor}
\begin{itemize}
\item PhD from Florida State University in Computer Science
\item Faculty Experience:
\begin{itemize}
\item College of Charleston: Adjunct 2013-2014
\end{itemize}
\item Work Experience:
\begin{itemize}
\item Google (2014): Android Bluetooth/Wi-Fi/Telephony
\item SPAWAR (2009-2014): Communication systems
\item DenimGroup (2004-2005): Start-up; web design and network security
\end{itemize}
\end{itemize}
\end{frame}

\begin{frame}{Syllabus}
Lets go over the syllabus...
\end{frame}

\section{Introduction}
\subsection{}

\begin{frame}{Overview}
\begin{itemize}
\item We begin by examining a very basic Java program and using it to explore some initial programming concepts
\item Chapter 1 focuses on
\begin{itemize}
\item Introducing the Java programming language
\item Describing the steps involved in program compilation and execution
\item Exploring the issues related to problem solving in general
\item Discussing the activities involved in the software development process
\item Presenting an overview of object-oriented principles
\end{itemize}
\end{itemize}
\end{frame}

\begin{frame}{Java Programming}
\begin{itemize}
\item A computer is made up of hardware and software
\item hardware - the physical, tangible pieces that support the computing effort
\item program - a series of instructions that the hardware executes one after another
\item Programs are sometimes called applications
\item software - consists of programs and the data those programs use
\end{itemize}
\end{frame}

\begin{frame}{Java Programming}
\begin{itemize}
\item A programming language specifies the words and symbols that we can use to write a program
\item A programming language employs a set of rules that dictate how the words and symbols can be put together to form valid program statements
\item The Java programming language was created by Sun Microsystems, Inc.
\item It was introduced in 1995 and its popularity grew quickly \url{http://www.tiobe.com/}
\end{itemize}
\end{frame}

\begin{frame}{Java Programming}
\begin{itemize}
\item In the Java programming language
\begin{itemize}
\item a program is made up of one or more classes
\item a class contains one or more methods
\item a method contains program statements
\end{itemize}
\item These terms will be explored in detail throughout the course
\item A Java application always contains a method called main
\end{itemize}
\end{frame}

\begin{frame}[fragile]{Example Java Program}
\begin{lstlisting}
//********************************************************************
//  Lincoln.java       Java Foundations
//
//  Demonstrates the basic structure of a Java application.
//********************************************************************

public class Lincoln {
   //-----------------------------------------------------------------
   //  Prints a presidential quote.
   //-----------------------------------------------------------------
   public static void main (String[] args) {
      System.out.println ("A quote by Abraham Lincoln:");

      System.out.println ("Whatever you are, be a good one.");
   }
}
\end{lstlisting}
\end{frame}

\begin{frame}[fragile]{Structure}
\begin{lstlisting}
// comments about the class

public class MyProgram { // <- class heading

    // class body

}

/* 
 * Note that comments can be place almost anywhere and you
 * may have single line and multi-line comments like in C.
 */
\end{lstlisting}
\end{frame}

\begin{frame}[fragile]{Structure}
\begin{lstlisting}
// comments about the class

public class MyProgram {

    // comments about a method

    public static void main(String[] args) { // <- method header

        // method body

    }
}
\end{lstlisting}
\end{frame}

\begin{frame}[fragile]{Comments}
\begin{itemize}
\item Comments in a program are called inline documentation
\item They should be included to explain the purpose of the program and describe processing steps
\item They do not affect how a program works
\item Java comments can take three forms
\end{itemize}
\begin{lstlisting}
// this comment runs to the end of the line

/*  this comment runs to the terminating
    symbol, even across line breaks        */

/** this is a javadoc comment   */
\end{lstlisting}
\end{frame}

\begin{frame}{Identifiers}
\begin{itemize}
\item \textit{Identifiers} are the words a programmer uses in a program
\begin{itemize}
\item can be made up of letters, digits, the underscore character, and the dollar sign
\item cannot begin with a digit
\end{itemize}
\item Java is \textit{case sensitive} - Total, total, and TOTAL are different identifiers
\item By convention, programmers use different case styles for different types of identifiers, such as
\begin{itemize}
\item title case for class names - Lincoln
\item upper case for constants - MAXIMUM
\end{itemize}
\end{itemize}
\end{frame}

\begin{frame}{Identifiers}
\begin{itemize}
\item Sometimes we choose identifiers ourselves when writing a program (such as Lincoln)
\item Sometimes we are using another programmer's code, so we use the identifiers that he or she chose (such as println)
\item Often we use special identifiers called reserved words that already have a predefined meaning in the language
\item A reserved word cannot be used in any other way
\end{itemize}
\end{frame}

\begin{frame}[fragile]{Reserved Words}
\begin{itemize}
\item The Java reserved words
\end{itemize}
\begin{columns}[c]
\column{0.25\textwidth}
\begin{lstlisting}
abstract
assert
boolean
break
byte
case
catch
char
class
const
continue
default
do
double
\end{lstlisting}

\column{0.25\textwidth}
\begin{lstlisting}
else
enum
extends
false
final
finally
float
for
goto
if
implements
import
instanceof
int
\end{lstlisting}
\column{0.25\textwidth}
\begin{lstlisting}
interface
long
native
new
null
package
private
protected
public
return
short
static
strictfp
super
\end{lstlisting}
\column{0.25\textwidth}
\begin{lstlisting}
switch
synchronized
this
throw
throws
transient
true
try
void
volatile
while
\end{lstlisting}
\end{columns}
\end{frame}

\begin{frame}{White Space}
\begin{itemize}
\item Spaces, blank lines, and tabs are called white space
\item White space is used to separate words and symbols in a program
\item Extra white space is ignored
\item A valid Java program can be formatted many ways
\item Programs should be formatted to enhance readability, using consistent indentation
\end{itemize}
\end{frame}

\begin{frame}[fragile]{Readability}
\begin{lstlisting}
//********************************************************************
//  Lincoln2.java       Java Foundations
//
//  Demonstrates a poorly formatted, though valid, program.
//********************************************************************

public class Lincoln2{public static void main(String[]args){
System.out.println("A quote by Abraham Lincoln:");
System.out.println("Whatever you are, be a good one.");}}
\end{lstlisting}
\end{frame}

\begin{frame}[fragile]{Readability}
\begin{lstlisting}
//********************************************************************
//  Lincoln3.java       Java Foundations
//
//  Demonstrates another valid program that is poorly formatted.
//********************************************************************

          public       class
     Lincoln3
   {
                 public
   static
        void
  main
        (
String
            []
    args                       ) 
  {
  System.out.println        (
"A quote by Abraham Lincoln:"          )
  ;        System.out.println
            (
       "Whatever you are, be a good one."
      )
  ;
}
          }
\end{lstlisting}
\end{frame}

\section{Program Development}
\subsection{}
\begin{frame}{}
\begin{itemize}
\item The mechanics of developing a program include several activities
\begin{itemize}
\item writing the program in a specific programming language (such as Java)
\item translating the program into a form that the computer can execute
\item investigating and fixing various types of errors that can occur
\end{itemize}
\item Software tools can be used to help with all parts of this process
\end{itemize}
\end{frame}

\begin{frame}{Language Levels}
\begin{itemize}
\item There are four main programming language levels
\begin{enumerate}
\item machine language
\item assembly language
\item high-level language
\item fourth-generation language
\end{enumerate}
\item Each type of CPU has its own specific machine language
\item The other levels were created to make it easier for a human being to read and write programs
\end{itemize}
\end{frame}

\begin{frame}{Programming Languages}
\begin{itemize}
\item Each type of CPU executes only a particular machine language
\item A program must be translated into machine language before it can be executed
\item A compiler is a software tool which translates source code into a specific target language
\item Often, that target language is the machine language for a particular CPU type
\item The Java approach is somewhat different
\end{itemize}
\end{frame}

\begin{frame}{Java Translation}
\begin{itemize}
\item The Java compiler translates Java source code into a special representation called bytecode
\item Java bytecode is not the machine language for any traditional CPU
\item Another software tool, called an interpreter, translates bytecode into machine language and executes it
\item Therefore the Java compiler is not tied to any particular machine
\item Java is considered to be architecture-neutral
\end{itemize}
\end{frame}

\begin{frame}{Java Translation}
\includegraphics[width=0.8\textwidth]{../imgs/java-compliation-overview.png}
\end{frame}

\begin{frame}{Development Environments}
\begin{itemize}
\item There are many programs that support the development of Java software, including
\begin{itemize}
\item Sun Java Development Kit (JDK)
\item Eclipse
\item NetBeans
\item BlueJ
\item Your favorite text editor (notepad, notepad++, vi, emacs, nano, etc...)
\end{itemize}
\item Though the details of these environments differ, the basic compilation and execution process is essentially the same
\end{itemize}
\end{frame}

\begin{frame}{Syntax and Semantics}
\begin{itemize}
\item The \textit{syntax} rules of a language define how we can put together symbols, reserved words, and identifiers to make a valid program
\item The \textit{semantics} of a program statement define what that statement means (its purpose or role in a program)
\item A program that is syntactically correct is not necessarily logically (semantically) correct
\item A program will always do what we tell it to do, not what we \underline{meant} to tell it to do
\end{itemize}
\end{frame}

\begin{frame}{Errors}
\begin{itemize}
\item A program can have three types of errors
\begin{itemize}
\item The compiler will find syntax errors and other basic problems (compile-time errors)
\begin{itemize}
\item If compile-time errors exist, an executable version of the program is not created
\end{itemize}
\item A problem can occur during program execution, such as trying to divide by zero, which causes a program to terminate abnormally (run-time errors)
\item A program may run, but produce incorrect results, perhaps using an incorrect formula (logical errors) 
\end{itemize}
\end{itemize}
\end{frame}

\begin{frame}{Basic Program Development}
\includegraphics[width=0.8\textwidth]{../imgs/novice-programming-cycle.png}
\begin{itemize}
\item When is it easiest to catch an error?
\item Software Engineering/process.
\end{itemize}
\end{frame}

\begin{frame}{Problem Solving}
\begin{itemize}
\item The purpose of writing a program is to solve a problem
\item Solving a problem consists of multiple activities
\begin{itemize}
\item understand the problem
\item design a solution
\item consider alternatives and refine the solution
\item implement the solution
\item test the solution $\leftarrow$ this is VITALLY important!!!
\end{itemize}
\item These activities are not purely linear - they overlap and interact
\end{itemize}
\end{frame}

\begin{frame}{Problem Solving}
\begin{itemize}
\item The key to designing a solution is breaking it down into manageable pieces
\item When writing software, we design separate pieces that are responsible for certain parts of the solution
\item An \textit{object-oriented} approach lends itself to this kind of solution decomposition
\item We will dissect our solutions into pieces called objects and classes
\end{itemize}
\end{frame}

\begin{frame}{Software Development Activities}
\begin{itemize}
\item Any proper software development effort consists of (at least) four basic development activities
\begin{itemize}
\item establishing the requirements
\item creating a design
\item implementing the design
\item testing
\end{itemize}
\item These steps also are never purely linear and often overlap and interact
\item Examples: 
\begin{itemize}
\item the test cases \textit{should} be considered during design
\item the quality of the code (which determines how well we test) \textit{should} be considered during requirements
\end{itemize}
\end{itemize}
\end{frame}

\begin{frame}{Software Development Activities}
\begin{itemize}
\item \textit{Software requirements} specify what a program must accomplish
\item Requirements are expressed in a document called a \textit{functional specification}
\item A \textit{software design} indicates how a program will accomplish its requirements
\item \textit{Implementation} is the process of writing the source code that will solve the problem
\item \textit{Testing} is the act of ensuring that a program will solve the intended problem given all of the constraints under which it must perform
\end{itemize}
\end{frame}

\section{Object-Oriented Programming}
\subsection{}
\begin{frame}{OOP}
\begin{itemize}
\item Java is an object-oriented programming (OOP) language
\item As the term implies, an object is a fundamental entity in a Java program
\item Objects can be used effectively to represent real-world entities
\item For instance, an object might represent a particular employee in a company
\item Each employee object handles the processing and data management related to that employee
\end{itemize}
\end{frame}

\begin{frame}{Objects}
\begin{itemize}
\item An object has
\begin{itemize}
\item \textit{state}  -  descriptive characteristics 
\item \textit{behaviors}  -  what it can do (or what can be done to it)
\end{itemize}
\item The state of a bank account includes its account number and its current balance
\item The behaviors associated with a bank account include the ability to make deposits and withdrawals
\item Note that the behavior of an object might change its state
\item Also Note: for those coming from C/C++, basically Java allows you to group your functions (behaviors) with a struct (state).
\end{itemize}
\end{frame}

\begin{frame}{Classes}
\begin{itemize}
\item An object is defined by a class
\item A class is the blueprint of an object
\item The class uses methods to define the behaviors of the object
\item The class that contains the main method of a Java program represents the entire program
\item A class represents a concept, and an object represents the embodiment of that concept
\item Multiple objects can be created (instantiated) from the same class
\end{itemize}
\end{frame}

\begin{frame}{Objects and Classes}
\includegraphics[width=0.7\textwidth]{../imgs/class-object.png}
\end{frame}

\begin{frame}{Inheritance}
\begin{itemize}
\item One class can be used to derive another via inheritance
\item Classes can be organized into hierarchies
\end{itemize}
\includegraphics[width=0.6\textwidth]{../imgs/inheritance.png} \\
Why would we do this?
\end{frame}

\section{Lab Assignment}
\subsection{}
\begin{frame}{}
\begin{itemize}
\item Login into your machine
\item Write and TEST a simple Hello World application
\item Submit via Blackboard
\item This is to test building a program on your machine and submitting via blackboard.
\item This is not graded.
\end{itemize}
\end{frame}

\begin{frame}{Possible Future Working tools}
\begin{itemize}
\item git
\item make
\item Android Development Tools
\end{itemize}
\end{frame}

%\begin{frame}{Event Based Processor}
%\begin{columns}[c]
%\column{0.25\textwidth}
%\begin{block}{OS driven Execution}
    %\includegraphics[width=1.0\textwidth]{imgs/normproc.png}
%\end{block}
%\column{0.25\textwidth}
%\begin{block}{Event Driven Execution}
    %\includegraphics[width=1.0\textwidth]{diagrams/eventproc.png}
%\end{block}
%\column{0.5\textwidth}
%\begin{itemize}
    %\item Normal execution: kernel/software driven
    %\item Event based : event driven
    %\item Performance monitoring is event based
    %\item Next task based on event not scheduled by kernel
    %\item 5.92 times speedup for control dominated programs
%\end{itemize}
%\end{columns}
%\end{frame}

\end{document}
